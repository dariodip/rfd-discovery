%\newcommand{\CLASSINPUTbaselinestretch}{1.0} % baselinestretch
%\newcommand{\CLASSINPUTinnersidemargin}{1in} % inner side margin
%\newcommand{\CLASSINPUToutersidemargin}{1in} % outer side margin
%\newcommand{\CLASSINPUTtoptextmargin}{1in}   % top text margin
%\newcommand{\CLASSINPUTbottomtextmargin}{1in}% bottom text margin
\documentclass[10pt,journal,compsoc]{IEEEtran}

\newcommand\MYhyperrefoptions{bookmarks=true,bookmarksnumbered=true,
pdfpagemode={UseOutlines},plainpages=false,pdfpagelabels=true,
colorlinks=true,linkcolor={black},citecolor={black},urlcolor={black},
pdftitle={Bare Demo of IEEEtran.cls for Computer Society Journals},%<!CHANGE!
pdfsubject={Typesetting},%<!CHANGE!
pdfauthor={Michael D. Shell},%<!CHANGE!
pdfkeywords={Computer Society, IEEEtran, journal, LaTeX, paper,
             template}}%<^!CHANGE!

% correct bad hyphenation here
\hyphenation{op-tical net-works semi-conduc-tor}


\begin{document}
\title{Bare Advanced Demo of IEEEtran.cls for\\ IEEE Computer Society Journals}

\author{Michael~Shell,~\IEEEmembership{Member,~IEEE,}
        John~Doe,~\IEEEmembership{Fellow,~OSA,}
        and~Jane~Doe,~\IEEEmembership{Life~Fellow,~IEEE}% <-this % stops a space
\IEEEcompsocitemizethanks{\IEEEcompsocthanksitem M. Shell was with the Department
of Electrical and Computer Engineering, Georgia Institute of Technology, Atlanta,
GA, 30332.\protect\\
% note need leading \protect in front of \\ to get a newline within \thanks as
% \\ is fragile and will error, could use \hfil\break instead.
E-mail: see http://www.michaelshell.org/contact.html
\IEEEcompsocthanksitem J. Doe and J. Doe are with Anonymous University.}% <-this % stops a space
\thanks{Manuscript received April 19, 2005; revised August 26, 2015.}}

\maketitle
% The paper headers
\markboth{Journal of \LaTeX\ Class Files,~Vol.~14, No.~8, August~2015}%
{Shell \MakeLowercase{\textit{et al.}}: Bare Advanced Demo of IEEEtran.cls for IEEE Computer Society Journals}

\section{Relaxed Functional Dependencies}
Una RFD $X \rightarrow Y$ si verifica su una relazione $r$ sse la distanza tra due tuple $t_1$ e $t_2$ in $r$ è inferiore di una certa soglia $\alpha_A$ su ogni attributo $A \in X$, allora la loro distanza è inferiore ad una soglia $\alpha_B$ su ogni attributo $I \in Y$. 
Data una relazione $r$, la scoperta di una RFD è il problema di trovare un minimal cover set di RFD che si verificano per $r$. Questo problema aggiunge ulteriore complessità al problema già complesso della scoperta delle dipendenze dei dati visto l'estremamente ampio spazio di ricerca dei possibili vincoli di similarità. Dunque è necessario trovare algoritmi efficienti in grado di estrarre RFD con vincoli di similarità significativi.

Se i vincoli di similarità delle soglie sono conosciuti per ogni attributo del dataset, scoprire gli RFD si riduce a trovare tutte le possibili dipendenze che soddisfano la seguente regola:
\begin{center}
le partizioni di tuple che sono simili sugli attributi LHS devono corrispondere a quelli che sono simili nel RHS.
\end{center}
Questo procedimento è simile a scoprire le FD in cui si mira a trovare partizioni di tuple che condividono lo stesso valore sull'RHS se loro condividono gli stessi valori nell'LHS. 

\subsection{Scoprire le RFD per una determinata soglia}
Esistono svariati metodi per scoprire le RFD data una determinata soglia $\epsilon$.

I metodi di discovery top-down effettuano una generazione di possibili FD livello per livello e controllano se questi si verificano.

\end{document}


