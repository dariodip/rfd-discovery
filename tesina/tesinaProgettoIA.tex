\documentclass[11pt]{article}

\usepackage[italian,english]{babel}
\usepackage{cite}

\begin{document}
\title{Discovery RFDs}

\author{Antonio~Altamura,
		Dario~Di Pasquale,
   	Mattia~Tomeo}% 
   	
 
\maketitle
\section{Relaxed Functional Dependencies}
Una RFD $X \rightarrow Y$ si verifica su una relazione $r$ se e solo se la distanza tra due tuple $t_1$ e $t_2$ in $r$ è inferiore di una certa soglia $\alpha_A$ su ogni attributo $A \in X$, allora la loro distanza è inferiore ad una soglia $\alpha_B$ su ogni attributo $I \in Y$. 
Data una relazione $r$, la scoperta di una RFD è il problema di trovare un \textit{minimal cover set} di RFD che si verificano per $r$. Questo problema aggiunge ulteriore complessità al problema già complesso della scoperta delle dipendenze dei dati visto l'estremamente ampio spazio di ricerca dei possibili vincoli di similarità. Dunque è necessario trovare algoritmi efficienti in grado di estrarre RFD con vincoli di similarità significativi

Se i vincoli di similarità delle soglie sono conosciuti per ogni attributo del dataset, scoprire gli RFD si riduce a trovare tutte le possibili dipendenze che soddisfano la seguente regola:
\begin{center}
le partizioni di tuple che sono simili sugli attributi LHS devono corrispondere a quelli che sono simili nel RHS.
\end{center}
Questo procedimento è simile a scoprire le FD in cui si mira a trovare partizioni di tuple che condividono lo stesso valore sull'RHS se loro condividono gli stessi valori nell'LHS. 

\subsection{Scoprire le RFD per una determinata soglia}
Esistono svariati metodi per scoprire le RFD data una determinata soglia $\epsilon$.

I metodi di discovery top-down effettuano una generazione di possibili FD livello per livello e controllano se questi si verificano.


\bibliographystyle{IEEEtran}
\bibliography{rdfsbib}
\end{document}
